\section{Modellering}

Vi har valgt å sammenligne negativ-binomialfordelingen og lognormalfordelingen
som grunnlag estimere antall dødsfall en uke fram i tid. Vi vil ta utgangspunkt
i en tidseriemodell som estimerer antall dødsfall i uke $t$ ut fra antall
dødsfall i uke $t-1$. 

En modell som er sentral i tidsserieanalyse er den autoregressive modellen (AR).
Modellen legger til grunn at det finnes korrelasjon mellom observasjonene i
en tidsserie. Ved å bruke korrelasjonen kan man prøve å estimere
sannsynligheten for fremtidige utfall av verdier i tidsserien. Antall verdier
fra tidsserien man inkluderer avgjør hvilken orden modellen har. I dette
tilfellet bruker vi observasjonene fra forrige tidssteg $t-1$ i modellen, så
det blir en 1. ordens autoregressiv modell (AR(1)). En standard AR(1)-modell
for kontinuerlig responsvariabel er formulert ved  

$$
Y_t|Y_{t-1} = a + bY_{t-1} + \epsilon_t,
$$

\noindent
der $a \in \mathbb{R}$, $0 < |b| <1$, og $\epsilon_t$ er feilen ved uke $t$.
Vider antar man at $\epsilon_t$ er en tilfeldig variable som er normalfordelt
med forventning 0 og standardavvik $\sigma$. For at AR(1)-modellen skal være
stasjonær kreves det at $b$-parameteren er begrenset til $|b| < 1$. At modellen
er stasjonær betyr at det finnes en marginal sannsynlighetsfordeling over $Y$
når $t \to \infty$. Hvis $|b| > 1$ vil modellen beskrive en tidsserie som er
ustabil og som kan vokse ubegrenset. 

Siden vi skal beskrive dødsfall, som er ikke negative heltall er det nødvendig
å gjøre noen modifikasjoner for å kunne bruke en AR(1)-modell. Det ene man kan
gjøre er å transformere responsvariabelen slik at den er representert som et
rasjonelt tall, vanligvis er det gjort ved å ta logaritmen til variabelen. En
annen metode er å finne forventningen betinget på observasjoner ved $t-1$ og så
finne en heltallsfordeling som beskriver observasjonene ved tid $t$ gitt
observasjonene ved tid $t-1$ \parencite[73]{weiss2018introduction}. For vårt eksempel vil den betingede forventningen
være 

$$
M_t|Y_{t-1} = a +bY_{t-1}.
$$

\noindent
Så lar man $Y_t ~ F(M_t)$ der $F$ er en heltallsfordeling. Hvis $F$ er
Poisson-fordelingen blir  

$$
Y_t|Y_{t-1} \sim \mathrm{Poi}(M_t|Y_{t-1}).
$$

\noindent



\subsection{Negativ binomial}

Den første modellen vi skal se på er en modell tilpasset heltallstidsserier med
overspredning. Modellen er beskrevet av \cite{xu2012model} og er en Negative
Binomial Dispersed Integer Auto Regressive Conditional Heteroscedasity-modell
(NB-DINARCH). Modellene blir referert til som NB-modellen i resten av oppgaven.
Ved å bruke negativ-binomialfordelingen har man mulighet til å modellere data
med overvekt av 0-verdier og overspredning. Dette er i motsetning til en modell
basert på Poisonnfordelingen, hvor den betingede variansen er lik den betingede
forventningen. La $Y_{i, t} \in \mathbb{N}_0$ være antall dødsfall i land $i$ i
uke $t$. Videre er den betingede forventningen

$$
M_{i, t}|Y_{i,t} = a_i + b_iY_{i,t-1},
$$

\noindent
hvor $a > 0$ og $0 < b < 1$. Da er modellen definert av

$$
Y_{i,t}|Y_{i, t-1} \sim 
\mathrm{NB}\left(c(M_{i,t}|Y_{i, t-1}), \frac{c}{c+1}\right),
$$

\noindent
hvor $c >0$. Forventningen til $Y_{i,t}|Y_{i,t-1}$ blir

$$
\mathrm{E}[Y_{i,t}|Y_{i,t-1}] = a + bY_{i,t-1}.
$$

\noindent
Variansen blir

$$
\mathrm{Var}(Y_{i,t}|Y_{i,t-1}) = \frac{(a + bY_{i,t-1})(c + 1)}{c}.
$$

\noindent
Parameteren $a$ blir en grunnrate i antall dødsfall pr uke. Parameteren $b$ er
en skaleringsfaktor som sier noe om hvor sterk sammenheng det er mellom dødsfall
i en uke til den neste. Parameteren $c$ påvirker spredningen til fordelingen.
Her ser man at forventningen er uavhengig spredningsparameteren $c$. Derimot
varierer variasjonen med forventningen. Dette gjør at modellen kan beskrive
populasjoner med både høye og lave forventninger. Vi kan og så se at variansen
går mot forventningen når $c \to \infty$, det vil si en Poisson-fordeling med
forventning og varians $M_{i,t}$. Modellen er dermed egnet til å beskrive
populasjoner med overspredning, da variansen kan anta verdier fra $M_{i,t}$
til $\infty$.

\subsection{Lognormal}

Den andre fordelingen vi skal bruke er lognormalfordelingen. Motivasjonen for å
bruke denne fordelingen i modelleringen er at man får redusert
heteroskedasiteten når verdiene er på den logaritmiske skalaen. En annen grunn
til å operere på logaritmisk skala er at modelleringen kan bli enklere. En anne
grunn er at ikke negative heltall er transformert til å være rasjonale tall. Da
kan modelleringen bli enklere og man kan bruke for eksempel en standard
AR(1)-modell.

Hvis $Y_{i,t}$ er antall dødsfall i land $i$ i uke $t$ definer vi den transformerte variabelen

$$
Z_{i,t} = \log(Y_{i,t} + 1).
$$
\noindent
Vi tar logaritmen av $Y_{i,t} + 1$ for at omgå at man tar $\log(0)$. Videre antar vi da at

$$
Z_{i,t}|Z_{i,t-1} = a_i + b_iZ_{i,t-1} + \epsilon_{i,t},
$$

\noindent
der $\epsilon_{i,t} \sim \mathrm{N}(0, \sigma_i)$.  Her er $a_i > 0$, $0 < b_i
< 1$ og $\sigma_i > 0$. Parameteren $a_i$ grunnraten til $Z_{i,t}$. Parameteren
$b_i$ er beskriver sammenhengen mellom $Z_{i, t-1}$ og $Z_{i,t}$. Parameteren
$\sigma_i$ beskriver standardavviket til $\epsilon_{i, t}$. Forventningen til
$Z_{i,t}|Z_{i,t-1}$ er

$$
Z_{i,t}|Z_{i,t-1} = a_i + b_iZ_{t-1}
$$

\noindent
og variansen er $\sigma_{i}^2$. Videre er forventningen til $Y_{i,t}|Z_{i,t-1}$

$$
\mathrm{E}[Y_{i,t}|Z_{i,t-1}] = e^{a_i +b_iZ_{t-1} \frac{\sigma_i^2}{2}} - 1.
$$

Variansen til $Y_{i,t}|Z_{i,t-1}$ er

$$
\mathrm{Var}(Y_t|Z_{i,t-1}) = (e^{\sigma_i^2}-1)e^{2(a_i+b_iZ_{i,t-1}) + \sigma_i^2}.
$$

En detalj å legge merke til  er at $\mathrm{E}[Y_{i,t}|Y_{i,t-1}]$ er avhengig
av spredningsparameteren $\sigma$. Selv om man på lognormalskalaen kan
modellere data med lav forventning og høy varians, vil forventningen på
originalskalaen øke med variansen. 

